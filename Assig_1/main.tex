\documentclass[journal,12pt,two column]{IEEEtran}
\usepackage{setspace}
\usepackage{enumitem}
\usepackage[cmex10]{amsmath}
\usepackage{amssymb}
\usepackage{tfrupee}
\usepackage{amsthm}
\usepackage[utf8]{inputenc}
\title{Assignment 1 (ICSE 2017)}
\author{Nitya Seshagiri Bhamidipaty (cs21btech11041)}
\newcommand{\solution}{\noindent \textbf{Solution: }}
\begin{document}
\maketitle
\begin{enumerate}
\item[\textbf{2 (c)}] Jaya borrowed \rupee50,000 for 2 years. The rates of interest for two successive years are 12\% and 15\% respectively. She repays \rupee33,000 at the end of the first year. Find the amount she must pay at the end of the second year to clear her debt.\\
\solution \\
The various parameters considered in this problem are listed in Table I.
\begin{table}[h!]
\label{table:table1}
\caption{}
\begin{tabular}{|c|c|p{4.8cm}|}
\hline
\textbf{Symbol} & \textbf{Value} & \textbf{Description}\\
\hline
$n$ & 2 & Total number of years over which the debt spans\\
\hline
$P$ & \rupee 50,000 & Principal\\
\hline
$R_1$ & $12 \% $ & Rate of interest for the $1^{\text{st}}$ year (per annum)\\
\hline
$R_2$ & $15 \% $ & Rate of interest for the $2^{\text{nd}}$ year (per annum)\\
\hline
$q_1$ & \rupee 33,000 & Amount paid at the end of $1^{\text{st}}$ year\\
\hline
$A_n$ & ? & Amount due at the end of $n^{\text{th}}$ year\\
\hline
\end{tabular}
\end{table}
\\
Amount due at the end of $1^{\text{st}}$ year
\begin{equation}
= P\left(1+\frac{R_1}{100}\right)
\end{equation}
The principal for the $2^{\text{nd}}$ year
\begin{equation}
     = P\left(1+\frac{R_1}{100}\right) - q_1
\end{equation}
Amount due at the end of $2^{\text{nd}}$ year
\begin{align}
    &= \left (P\left(1+\frac{R_1}{100}\right) - q_1\right)\left(1 + \frac{R_2}{100}\right)
\end{align}
\begin{multline}
      = P\left(1+\frac{R_1}{100}\right)\left(1+\frac{R_2}{100}\right) \\ - q_1\left(1+\frac{R_2}{100}\right)\label{eq:4}
\end{multline}
$\because n = 2$,
we can directly substitute the values in eq\eqref{eq:4} to get the desired value\\
Also note that in eq\eqref{eq:4}:
\begin{enumerate}
        \item The left term corresponds to $P$ + the interest on $P$
        \item The right term corresponds to $q_1$ + interest on $q_1$.
\end{enumerate}
So for $n$ years, 
\begin{multline}
\label{eq:5}
A_n = P\prod_{k = 1}^{n}{\left(1+\frac{R_k}{100}\right)} -\\ \sum_{i=1}^{n-1}{\left(q_i\prod_{l=i+1}^{n}{\left(1+\frac{R_l}{100}\right)}\right)}
\end{multline}
\begin{table}[h!]
\label{table:table2}
   \caption{Variables of the eq\eqref{eq:5}}
    \centering
    \begin{tabular}{|c|p{5.4cm}|}
    \hline
        \textbf{Symbol} & \textbf{Description}  \\
    \hline
        $n$ & Total number of years the loan spans over\\
    \hline
        $P$ & Principal\\
    \hline
        $R_1, R_2, \ldots, R_n$ &  The rate of interests (per annum) corresponding to the $1^\text{st}$, $2^\text{nd}$, \ldots, $n^\text{th}$ year respectively\\
    \hline
        $q_i$ & Amount paid at the end of the  $i^\text{th}$ year ( $\forall i\in \{1,2,\ldots,n-1\})$\\
    \hline
    \end{tabular}
    \label{tab:my_label}
\end{table}\\
Table II describes all the variables in eq\eqref{eq:5}
Substituting the values of
$n$, $P$, $R_1$, $R_2$, $q_1$ in eq\eqref{eq:5} we get
 \begin{multline}
    A_2 = 50000 \prod_{k = 1}^{2}  {\left(  1+\frac{R_k}{100}     \right)} \\ - \sum_{i=1}^{1}{\left(q_i\prod_{l=i+1}^{2}{\left(1+\frac{R_l}{100}\right)}\right)}
\end{multline}
\begin{multline}
    A_2 =50000\left(1+\frac{R_1}{100}\right)\left(1+\frac{R_2}{100}\right)\\ - q_1\left(1+\frac{R_2}{100}\right)
\end{multline}
\begin{multline}
       A_2  =50000\left(1+\frac{12}{100}\right)\left(1+\frac{15}{100}\right) \\ - 33000\left(1+\frac{15}{100}\right)
\end{multline}
\begin{align}
   \implies A_2 = 26450
\end{align}
Hence, Jaya must pay \rupee26,450 at the end of the second year to clear her debt.
\end{enumerate}
\end{document}
