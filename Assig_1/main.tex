\documentclass[journal,12pt,two column]{IEEEtran}
\usepackage{setspace}
\usepackage{enumitem}
\usepackage[cmex10]{amsmath}
\usepackage{tfrupee}
\usepackage{amsthm}
\usepackage[utf8]{inputenc}
\title{Assignment 1 (ICSE 2017)}
\author{Nitya Seshagiri Bhamidipaty (cs21btech11041)}
\newcommand{\solution}{\noindent \textbf{Solution: }}
\begin{document}
\maketitle
\begin{enumerate}
\item[\textbf{2 (c)}] Jaya borrowed \rupee50,000 for 2 years. The rates of interest for two successive years are 12\% and 15\% respectively. She repays \rupee33,000 at the end of the first year. Find the amount she must pay at the end of the second year to clear her debt.\\
\solution \\
Total number of years over which the debt spans = $n$\\
Principal amount =  \rupee$P$\\
Rate of interest for the $1^{\text{st}}$ year =  $R_1\%$ p.a
Rate of interest for the $2^{\text{nd}}$ year =  $R_2\%$ p.a
Amount paid at the end of $1^{\text{st}}$ year = \rupee $q_1$\\ 
Amount due at the end of $1^{\text{st}}$ year
\begin{equation}
     = P\left(1+\frac{R_1}{100}\right)
\end{equation}
The principal for the $2^{\text{nd}}$ year
\begin{equation}
     = P\left(1+\frac{R_1}{100}\right) - q_1
\end{equation}
Amount due at the end of $2^{\text{nd}}$ year
\begin{align}
    &= \left (P\left(1+\frac{R_1}{100}\right) - q_1\right)\left(1 + \frac{R_2}{100}\right)
    \\
    &= P\left(1+\frac{R_1}{100}\right)\left(1+\frac{R_2}{100}\right) - q_1\left(1+\frac{R_2}{100}\right)\label{eq:4}
\end{align}
We can directly substitute the values in eq\eqref{eq:4} to get the desired value.\\
Note that in eq\eqref{eq:4}:
    \begin{enumerate}
        \item The left term corresponds to $P$ + the interest on $P$ (the principal).
        \item The right term corresponds to $q_1$ (the amount paid in the previous year) + interest on $q_1$.
\end{enumerate}
Similarly generalising over $n$ years (instead of 2 years), 
we can say that the amount due at the end of the $n^{\text{th}}$ year is 
\begin{multline}
 = P\prod_{k = 1}^{n}{\left(1+\frac{R_k}{100}\right)} -\\ \sum_{i=1}^{n-1}{\left(q_i\prod_{l=i+1}^{n}{\left(1+\frac{R_l}{100}\right)}\right)}
\end{multline}
%
Here,
 the loan of \rupee$P$ spans over $n$ years. The rate of interests (per annum) corresponding to the $1^\text{st}$, $2^\text{nd}$, \ldots, $n^\text{th}$ year being $R_1, R_2, \ldots, R_n$  respectively. \rupee$q_i$ is the amount paid at the end of the  $i^\text{th}$ year ( $\forall i\in \{1,2,\ldots,n-1\})$\\
substituting the values,\\
$n = 2$, $P = 50000$, $R_1=12$, $R_2=15$, ${q_1=33000}$
 \begin{multline*}
     = 50000 \prod_{k = 1}^{2}  {\left(  1+\frac{R_k}{100}     \right)} \\ - \sum_{i=1}^{1}{\left(q_i\prod_{l=i+1}^{2}{\left(1+\frac{R_l}{100}\right)}\right)}
\end{multline*}
\begin{multline*}
     =50000\left(1+\frac{R_1}{100}\right)\left(1+\frac{R_2}{100}\right)\\ - q_1\left(1+\frac{R_2}{100}\right)
\end{multline*}
\begin{multline*}
         =50000\left(1+\frac{12}{100}\right)\left(1+\frac{15}{100}\right) \\ - 33000\left(1+\frac{15}{100}\right)
\end{multline*}
\hspace{5mm}$= 26450$\\
Hence, Jaya must pay \rupee26,450 at the end of the second year to clear her debt.
\end{enumerate}
\end{document}
