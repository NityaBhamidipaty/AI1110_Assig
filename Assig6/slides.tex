\documentclass{beamer}
\usepackage{amssymb}
\usepackage{amsmath}
\usetheme{CambridgeUS}

\title{Assignment 6} 
\author{Nitya Seshagiri Bhamidipaty (CS21BTECH11041)}
\date{\today}
\logo{\large \LaTeX{}}

\def\inputGnumericTable{}

\usepackage[latin1]{inputenc}                                 
\usepackage{color}                                            
\usepackage{array}                                            
\usepackage{longtable}                                        
\usepackage{calc}                                             
\usepackage{multirow}                                         
\usepackage{hhline}                                           
\usepackage{ifthen}
\usepackage{caption} 
%\captionsetup[table]{skip=3pt}  
\providecommand{\brak}[1]{\ensuremath{\left(#1\right)}}
\providecommand{\pr}[1]{\ensuremath{\Pr\left(#1\right)}}
\providecommand{\cbrak}[1]{\ensuremath{\left\{#1\right\}}}
\newcommand*{\permcomb}[4][0mu]{{{}^{#3}\mkern#1#2_{#4}}}
\newcommand*{\perm}[1][-3mu]{\permcomb[#1]{P}}
\newcommand*{\comb}[1][-1mu]{\permcomb[#1]{C}}

\begin{document}
\begin{frame}
    \titlepage 
\end{frame}
\logo{}
\begin{frame}{Outline}{NCERT Class 12 Example 29}
    \tableofcontents
\end{frame}
\section{Question}
\begin{frame}{Problem}
Q) Two cards are drawn simultaneously (or successively without replacement) from a well shuffled pack of 52 cards. Find the mean, variance and standard deviation of the number of kings.
\end{frame}
\section{Solution}
\begin{frame}{Solution}
    Let the random variable $X \in\{0,1,2\}$ denote 'The number of kings in a draw of two cards'.\\
    Then,
    \begin{align}
        \pr{X = 0} &= \frac{\comb{48}{2}}{\comb{52}{2}} \\
        &= \frac{\frac{48!}{2!(48-2)!}}{\frac{52!}{2!(52-2)!}}\\
        &= \frac{188}{221}
    \end{align}
    \end{frame}
\begin{frame}{Solution}
    \begin{align}
        \pr{X = 1} &= \frac{\comb{4}{1}\comb{48}{1}}{\comb{52}{2}}\\
                    &= \frac{32}{221}
    \end{align}
    \begin{align}
        \pr{X = 2} &= \frac{\comb{4}{2}}{\comb{52}{2}}\\
                    &= \frac{1}{221}
    \end{align}
\end{frame}

\begin{frame}{Solution}
    \begin{table}[h!]
        \centering
        \input{tables/table}
        \caption{Probability Distribution of X}
        \label{tab:table1}
    \end{table}
\end{frame}

\begin{frame}{Graph}
\begin{figure}
    \centering
    \includegraphics[width = 0.7\textwidth]{figures/plot}
    \caption{PMF and CDF of $X$}
    \label{fig:figure1}
\end{figure}
\end{frame}
\begin{frame}{Mean of $X$}
\begin{align}
\text{Mean of $X$} &=\\
E(X) &= \sum_{i=1}^nx_ip(x_i)\\
&= 0\times\frac{188}{221}+1\times\frac{32}{221}+2\times\frac{1}{221}\\
&= \frac{34}{221}
\end{align}
\end{frame}
\begin{frame}{Variance}
    \begin{align}
    E(X^2) &= \sum_{i=1}^{n}x_i^2p(x_i)\\
    &= 0^2\times\frac{188}{221}+1^2\times\frac{32}{221}+2^2\times\frac{1}{221}\\
    &= \frac{36}{221}
\end{align}
\begin{align}
    \sigma_x^2 &= E(X^2) - E(X)^2\\
    &= \frac{36}{221} - \brak{\frac{34}{221}}^2 \\
    &= \frac{6800}{(221)^2}
\end{align}
\end{frame}
\begin{frame}{Standard Deviation}
    \begin{align}
    \sigma_x &= \sqrt{Var(X)} \\
    &= \frac{\sqrt{6800}}{221}\\
    &= 0.37
\end{align}
\end{frame}
\section{Code}
\begin{frame}{Code}
Download all codes from
\begin{block}{Python}
         Download python code from - \href{https://github.com/NityaBhamidipaty/AI1110_Assig/tree/main/Assig6/codes}{Python}
    \end{block}
\end{frame}
\end{document}